\documentclass[Bachelorarbeit.tex]{subfiles}
\begin{document}
% how to prove that the final implementation works
\chapter{Test scenario}
\section{Target}
The target of the implementation is to use the video from a web cam to test the implemented color based face detection. The implemented code should as simple as possible, run the code on a Raspberry Pi.

\section{Implementation steps}
Following steps will be done to test if the implemented solution is real-time capable.

\subsection{Color based face detection on a picture}
The first step is to test the algorithms on different pictures (private pictures and pictures from \href{https://commons.wikimedia.org/wiki/Main_Page}{WIKIMEDIA COMMONS}). For this following steps are scheduled:
\begin{enumerate}
\item Transform picture into the YCbCr color space.
\item Find suitable threshold ranges for the YCbCr.
\item Make Thresholding on the YCbCr to get a binary picture.
\item Detecting faces out of skin regions.
\item Draw boxes to identify the faces on the picture.
\end{enumerate}

\subsection{Color based face detection on a video}\label{CbVidoe}
Use a web cam and test the implemented color based algorithm.

\subsection{Color based face detection on a Raspberry pi}
In this project a Raspberry Pi Model B2 and the Raspberry Pi camera module V? will be used.

\subsubsection{Simulink Model Image}
The first step is to transform the developed algorithm from MATLAB to Simulink. The \textit{Image Processing Toolbox} provides boxes which covers some needed features. This boxes will be used to achieve a high performance (assumption is that the toolbox from Mathworks is as efficient as possible). 

\subsubsection{Simulink Model on host PC and Hardware from Raspberry Pi}
In the next step a Simulink model should be created which can be deployed on the Raspberry Pi and check the results (original camera video frames and face detected video frames) with figures/Displays on the host PC.

\subsubsection{Simulink Model on Raspberry Pi}
The final step is to use the developed Simulink model to run it in a Standalone version on the Raspberry Pi. To see the result the modified video (including face detection) will be streamed over the internal HDMI port to a Display or a beamer. 

\end{document}
