\documentclass[Bachelorarbeit.tex]{subfiles}
\begin{document}
\chapter{Results}

\section{Findings}
\begin{itemize}
\item Resolution of the image/video must be defined, because the resolution has an influence to the amount of blob pixels (area property) and the computation time.
\item Thresholding values are various for different camera types. For every camera the thresholds must be found (Webcam has other thresholds than the raspberry pi camera module).
\item Lighting conditions can affect the algorithm – e.g. shadows (also depending on the used camera).
\item Hands and other skin regions are often interpreted as faces, without any additional filtering (like concerning the ratio of the rectangular).
\item Adjacent faces melt to one big rectangle.
\item Simulink blocks don't support all region properties (e.g. no euler number)in contrast to the Matlab algorithms. For some applications/ situations additional morphological operations are necessary.

\end{itemize}

\section{Improvements}
\begin{itemize}
\item To increase the speed the Raspberry Pi 3, or other powerful embedded hardware, instead of the Raspberry Pi 2 can be used.
\item To increase the speed, the algorithm should be programmed in PYTHON.
\item To get rid of the resolution problem (different amount of blob analysis), a function which calculates the resolution depending parameters (e.g. Area) can be implemented.
\end{itemize}
\end{document}